% slides-intro.tex
% MongoDB Course (in Russian)
% Author: Evgeny Simonenko <easimonenko@mail.ru>
% License: CC BY-ND 4.0

\documentclass[11pt]{beamer}

\usepackage[utf8x]{inputenc}
\usepackage[OT1]{fontenc}
\usepackage[english, russian]{babel}
\usepackage{graphicx}
\usetheme{Boadilla}

\begin{document}

\author{Симоненко Е.А. \\ \texttt{easimonenko@mail.ru}}
\title{MongoDB}
\subtitle{Введение}
\date{2018}
\setbeamertemplate{navigation symbols}{}

\begin{frame}
\titlepage
\end{frame}

\begin{frame}
\frametitle{Содержание}
\tableofcontents
\end{frame}

\section{История MongoDB}

\begin{frame}
\frametitle{История MongoDB}
\begin{itemize}
	\item 2007: компания 10gen из Нью-Йорка начала разрабатывать MongoDB как 
	продукт PAAS (Platform as a Service).
	\item 2009: MongoDB выпущена как открытый проект, а компания начала его 
	поддерживать.
	\item 2013: компания переименована в MongoDB Inc.
\end{itemize}
\end{frame}

\section{MongoDB vs. SQL Databases}

\begin{frame}
\frametitle{MongoDB vs. SQL Databases}
\begin{columns}
	\column{0.5\textwidth}
	\begin{block}{MongoDB}
		\begin{itemize}
			\item схема отсутствует
			\item язык запросов в форме JavaScript
		\end{itemize}
	\end{block}
	
	\column{0.5\textwidth}
	\begin{block}{SQL Databases}
		\begin{itemize}
			\item схема обязательна
			\item стандартизованный SQL
		\end{itemize}
	\end{block}
\end{columns}
\end{frame}

\section{Возможности и особенности MongoDB}

\begin{frame}
\frametitle{Возможности MongoDB}
\begin{itemize}
	\item документно-ориентированная СУБД
	\item отсутствие схемы
	\item встроенная поддержка JavaScript
	\item свободно-распространяемая (GNU AGPL)
\end{itemize}
\end{frame}

\section{Библиография}

\begin{frame}
\frametitle{Библиография}
\begin{thebibliography}{1}
  \bibitem[Бэнкер]{BankerRU} Бэнкер К. MongoDB в действии. -- Пер. с англ. -- 
  М.: ДМК Пресс, 2014. -- 394 с.
  \bibitem[Banker]{BankerEN} Banker, Kyle. MongoDB in Action. -- Manning, 2012. 
  -- 287 pp.
\end{thebibliography}
\end{frame}

\section{Ссылки}

\begin{frame}
\frametitle{Ссылки}
\begin{itemize}
	\item https://www.mongodb.com/
	\item https://github.com/mongodb/mongo
\end{itemize}
\end{frame}

\end{document}
